\documentclass[11pt,twocolumn]{article}
\usepackage[utf8]{inputenc}
\usepackage{amsmath}
\usepackage{amsfonts}
\usepackage{amssymb}
\usepackage{graphicx}
\usepackage[utf8]{inputenc}
\usepackage{amsmath}

\graphicspath{}

\author{
  \texttt{Sebastian Bångerius}
  \and
  \texttt{Villiam Rydfalk}
}

\begin{document}
\pagenumbering{gobble}

\title{Multi-path throughput in mobile networks}
\maketitle

\cleardoublepage


\section{Abstract}

We have made a project about the throughput in multipath networks. Specifically in wifi and bluetooth networks. By analysing and comparing these we have made a few conclusions about strengths and weaknessess for both. We have done a simulation of some scenarios based on the real world and what might happen in a network. We had some hypothesis about what should happen in these scenarios and discussed possible solutions. Once we conducted the simulations we revised our hypothesis and made an analysis of the results. We came to a conclusion and discussed other possible scenarios. We have also made some theories about when and where the different methods of networkrouting should be used and a few examples of implementations.

\section{Introduction}

The way we transfer information varies a lot between places and situations. Some techniques work better in certain environments, while they encounter complications in others. People in different countries have different interests, standards and financial potential. Because of all these circumstances affecting mobile networks, it is important to actually study how information flows through them. For instance; in areas with high population density there might be easier to rout data between end users towards an access point, while in sparsely populated areas it might be better with one cellular tower (or even a satellite) for connection. In a lot of central and southern African countries people have access to a phone but no electricity, in such an environment power saving might be a key feature of data transfer, and some power consuming routing options are excluded.

There are certainly many ways to transfer information electronically, however this report is about the different flows of information through different mobile networks for both voice and data. We will evaluate strengths and weaknesses for various techniques including WiFi, GSM, 3G and Bluetooth networks. We will discuss where and how these techniques work best, and how information is distributed through the networks under these conditions. We will in depth present a few of the previously mentioned technologies so they can be carefully evaluated and compared when it comes to routing.


\section{Method}

We want to evaluate different methods of pathing in a network to get good throughput. If we are to do any tests on a larget scale it will be a lot of complex work unless we do a simulation. Since we want to do tests based on multipathing the simulation program has to include a few things. 

First of all we need a few nodes that can connect in a way we want, and they have to be able to communicate with a given quantity of data that we can mesure. We also need to be able to split this data up in any way we want between several paths. In these paths we have to be able to set different factors like noise, other traffic and just about anyting that will interfere with our packets in the real world.

\section{Time plan}
\begin{description}
\item[Week 38]
Acquire information about how packets flow through different network types. Get a more detailed plan.
\item[Week 39]
Write ~2 pages. Find and learn tools for network simulation. Complete milestone 2
\item[Week 40]
Write ~2 more pages (so ~4 in total). Make first trial simulations
\item[Week 41]
Write ~3 more pages (so ~7 in total). Make sharp simulation. Draw conclusions. Make beautiful diagrams in Excel. Drool over said diagrams.
\item[Week 42]
Seminar and finish report
\end{description}



\end{document}