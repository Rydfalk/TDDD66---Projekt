\documentclass[11pt,twocolumn]{article}
\usepackage[utf8]{inputenc}
\usepackage{graphicx}



\graphicspath{}

\author{
  \texttt{Sebastian Bångerius}
  \and
  \texttt{Villiam Rydfalk}
}

\begin{document}
\pagenumbering{gobble}

\title{Multi-path throughput in mobile networks}
\maketitle

\cleardoublepage


\section{Abstract}

In this project we evaluate and draw some conclusions about the use and efficiency of TCP with multipathing, which is a new protocol in progress often referred to as MPTCP. We will only have data from an LTE network, but will discuss the use of this in other networks as it is a strength of the protocol to use many paths, possibly over many different types of networks. We aspired to do our own simulations, but as the time was short and the programs we found  were too complex and our knowledge of coding C++ was limited we could not complete any of our goals. We will include a discussion of our, even though limited, experience with these simulators as well as the strengths and weaknesses of two of them. So instead of using our own simulations as grounds we had to use results from another study. \cite{MPTCP-LTE} We analysed the results and made some hypotheses about what we were expecting. Once we'd analysed their method and looked over their results we made our own conclusions and posed a few new questions that would be interesting to investigate in future studies.



\section{Introduction}

The way we transfer information varies a lot between places and situations. Some techniques work better in certain environments, while they encounter complications in others. Because of all the circumstances affecting mobile networks, it is important to actually study how information flows through them. Especially in these times when our mobile devices tend to have more ways to connect to the internet (WiFi, LTE, 3G, Bluetooth etc.) and thus more network adapters, the idea of having these adapters working better together becomes more interesting.

Even though it is old, TCP is the most widely used protocol for data transferring, but with a growing internet and the number of users rising extremely fast the need for a more effective service is needed. Specifically with mobile devices like phones that use many different types of networks, there is a lot of potential for improvement. They can establish connections to the Internet via various interfaces such as 3G, WiFi, LTE and Bluetooth. The implementation of TCP over multiple channels is something that was discussed during the 10th International IFIP TC 6 Networking conference in Valencia in 2011 \cite{RFC6824}. It was argued that as our devices have so many different interfaces we need to update the original protocol that was introduced when we used only one interface for the transport layer in the protocol stack. And multiple channels for utilization is something we, in this report, hope to be able to at least argue for and make a few assumptions about it in the end.

\section{MPTCP Overview}
The Internet Engineering Task Force (IETF) is working on MPTCP. RFC 6824 is experimental and has been in progress since January 2013. It is well worked out and is not far from ready to be put to industrial use. Even so, there is no real implementation of it outside the experimental phase, so there are no official finished simulators or modules of MPTCP yet. There are a few studies made outside IETF based on what the RFC has stated and they have made some simulations that are publicly available, but they are not easily implemented unless a lot of work is put down.

MPTCP uses the same method for sending and receiving packets as TCP. The difference is that it can do so with multiple connections simultaneously. If both the client and server have several addresses linked to it they can set up several connections between themselves using a different address for each connection. These extra connections are usually called additional subflow setups, in contrast to the initial connection setup. MPTCP identifies multiple addresses and for each pair it sets up a connection. Each of these connections then works as a normal TCP connection starting with the normal syn, syn-ack and ack as handshake and it uses the same method of slow start and congestion control as normal TCP.

MPTCP is also supposed to be able to open subflows on different TCP-ports. The benefits of this  aren't as obvious as in the previously mentioned scenario (of different network addresses) as chances are the packets will end up on the same route anyway.

\subsection{Hypothesis}

The obvious expectation is of course that the throughput for MPTCP users will be higher than that of TCP users. A more effective use of the network where there would be unused paths. There are however a few drawbacks we can expect. First of all there should be a more spread out traffic in the network, which is both good and bad. It might spread out the traffic making a single path less burdened, but it might also make a slow lane used even more than it would if the users used normal TCP as it would be ruled out as ineffective. Another aspect is that it would probably consume more energy as it might use several different interfaces. 



\section{Simulators and their bottlenecks}

\subparagraph{NS-3}

NS-3 is a network simulator mainly for linux clients, with a few experimental versions for Windows. We used a virtual machine (Oracle VM VirtualBox) to set up Linux and then we followed a installation guide\cite{Guide} to install it. Once properly set up we had access to a lot of different modules and could run a number of pre-installed tests. 

\subparagraph{OMNeT++}

OMNeT++ is another network simulator that is quite different to NS-3. The main difference is the compatibility with Windows and the Eclipse based GUI. The drag and drop function of making networks is very useful and a lot easier to understand. OMNeT++ also has a very graphical focus with many very clear animations and support for google earth plugins.

\subsection{Our experience}

Unfortunately, we had a lot of problems dealing with the software and setting up a simulation. We had a good idea of what we wanted to do. A very simple network with a client, a server and 2 different paths between them with different quality. We never had a chance to make this simulation ourselves because of a few things. Just getting the virtual machine of Linux going and installing NS-3 on that virtual machine was a hassle. When something went wrong we had to redo the whole process since we have had very limited experience with Linux before. Once we had NS-3 up and running we encountered the next problem. There are no finished modules of MPTCP, and no real way to actually measure MPTCP throughput unless you write a lot of C++ code. And since we have no experience with C++ nor making NS-3 modules, this was a big setback. After doing a few of the tests that are included in the installation package we searched around if someone had tried to do the same thing. That is when we found OMNeT++. Since OMNeT++ had a much easier to use GUI (based on Eclipse that we have used a lot) we could quickly install and try some of it's functions, but like NS-3 it had no MPTCP implemented. 

A summary of the strengths and weaknesses would be that NS-3 is the older and most reworked of the two, and thus has a giant bank of different modules. And in contrast to OMNeT++, NS-3 has a much better output of measurements of, for example, bandwidth, packets sent and network speeds. What NS-3 lacks is a easy to use interface. There are some graphical interfaces that are downloadable, but it is still a hassle to even install modules on it unless you know exactly what you are doing. OMNeT++ on the other hand has a very nice GUI and is much easier to use, specifically to set up your own simulations in as it doesn't require as much programming. OMNeT++ also implements many plugins like google earth which makes it a bit more lucrative as it can show large networks in a realistic fashion.

Even though neither software had what we needed, we've gained a lot of insight. Next time we are doing a similar project we will have both these programs in the back of our head. And since they both have strengths and weaknesses they probably cover a lot of our needs.

\section{Litterature-based Results}

Since we were unable to conduct our own experiment we used the results from another report that made the experiment we had in mind (with some slight variations). In this experiment there is a client with two LTE interfaces and a server with two ethernet interfaces. Between them there are four connections. See figure 1.

\begin{figure}[ht]
\begin{center}
\includegraphics[scale=0.26]{figure_1}
\caption{Result table of test 1}
\end{center}
\end{figure}

A program called Iperf was used to monitor the TCP connections, and a variety of different programs were used to collect, analyse and structure the measurements of the network. But the basics of the network is that the LTE interface ppp1 has a better signal than ppp0. So by having this experiment set up the goal was to verify the efficiency of MPTCP in an LTE mobile network environment. There were three separate tests done. The first simulation was a TCP connection over a single path where they test the throughput and RTT to have something to compare with. The results were as follows in figure 2 and figure 4.

\begin{figure}[ht]
\begin{center}
\includegraphics[scale=0.5]{table_1}
\caption{Result table of test 1}
\end{center}
\end{figure}
\begin{figure}[ht]
\begin{center}
\includegraphics[scale=0.5]{graph_1}
\caption{Result graph of test 1}
\end{center}
\end{figure}

This test shows the maximum throughput without MPTCP. As we do not have data of other types of networks or any experiance with testing other types of networks, this will only be used for comparison. In the source they make some statements about the results confirms the behaviour of LTE networks as they are supposed to have lower delays than, for example, EDGE. 

In the second experiment the MPTCP protocol was used, which generated very different results. We can verify that the channel 0 had a worse signal than channel 1, but managed to maintain a relatively high throughput even though the RTT was high. So if we compare figure 3 and 5 we can see clearly that the MPTCP has a much higher total throughput than a single path TCP connection.

\begin{figure}[ht]
\begin{center}
\includegraphics[scale=0.6]{table_2}
\caption{Result table of test 2}
\end{center}
\end{figure}
\begin{figure}[ht]
\begin{center}
\includegraphics[scale=0.5]{graph_2}
\caption{Result graph of test 2}
\end{center}
\end{figure}
\begin{figure}[ht]
\begin{center}
\includegraphics[scale=0.5]{graph_3}
\caption{Result graph of test 2}
\end{center}
\end{figure}


\section{Discussion and Conclusion}

It is fairly obvious that MPTCP had a better throughput than the single connection TCP. The total throughput of the single path TCP was around 3 Mbps while the total throughput of MPTCP was around 7. Almost the double. Even though this is a very small simulation it does show the main strength of MPTCP, the ability to use more of the available paths than normal TCP. If we were to add a larger network we would presume the results to be very similar, maybe even better if there are a higher number of possible paths. Since the normal TCP is restricted to one path it just cannot compete on that type of problems if there are multiple available paths. If there was additional traffic with the chance of congestion and a higher risk of dropped packets the MPTCP would also have an advantage as it can use paths less travelled and thus be more effective. So if all devices would use such a method the efficient use of a network would surely rise, and the throughput would hopefully be higher as a result. Of course this would mean there would be a more evenly spread of traffic on almost every path and thus make it more likely to have collisions on even a long path that wouldn't be used with normal TCP.

There are some problems with MPTCP of course. In a presentation by IETF there are some pointers to problems with MPTCP.\cite{IETF-Probs}. They show that in some cases MPTCP could penalize TCP users. While the bottleneck for MPTCP users is at the server the TCP users often get a bottleneck in their own network. This means that MPTCP users make TCP users less effective without gaining throughput. As we have hinted before MPTCP might even make it harder for other MPTCP users as it would mean that everyone in the network gets a lowered throughput since a single user might send data through several bottlenecks at once, and thus making it harder for everyone to get data through the network.

So in conclusion what we can say about MPTCP is that it would be a good upgrade if we can first take care of the problems that it would cause for already existing users.

\newpage

\begin{thebibliography}{9}

\bibitem{MPTCP-LTE}
Ondrej VONDROUS , Peter MACEJKO , Zbynek KOCUR \\
\emph{Multipath TCP in LTE Networks}\\
Czech Technical University in Prague, 2014

\bibitem{VALENCIA}
BARRE, S., Ch. PAASCH and O. BONAVENTURE.\\
\emph{MultiPath TCP: From Theory to Practice.}\\
In: 10th International IFIP TC 6 Networking
Conference. Valencia: Springer, 2011

\bibitem{RFC6824}
RFC 6824.  \\
\emph{TCP Extensions for Multipath Operation with Multiple Addresses} \\ 
https://tools.ietf.org/html/rfc6824.\\ IETF, 2013.

\bibitem{IETF-Probs}
R. Khalili\\
\emph{Performance issues with MPTCP}\\
https://www.ietf.org/proceedings/84/slides/slides-84-mptcp-4.pdf

\bibitem{Comparison}
Murat Miran Köksal\\
\emph{A Survey of Network Simulators Supporting Wireless Networks} \\
October 22, 2008





\end{thebibliography}
\end{document}